%% Emacshandbuch in deutsch
%%
%% Vorwort
%%
%% $Revision: 1.2 $
%% $Id: vorwort.tex,v 1.2 2004/04/20 20:35:23 spica Exp $
%%
%%%%%%%%%%%%%%%%%%%%%%%%%%%%%%%%%%%%%%%%%%%%%%%%%%
%
% vorwort.tex -> Preface File
%
% Copyright (C) 2004 Joerg Meinhold 
%
% This document is free software; you can redistribute it and/or modify
% it under the terms of the GNU General Public License as published
% by the Free Software Foundation; either version 2 of the License.
%
% This document is distributed in the hope that it will be useful, but
% WITHOUT ANY WARRANTY; without even the implied warranty of MERCHANTABILITY
% or FITNESS FOR A PARTICULAR PURPOSE. See the GNU General Public License
% for more details.
%
% You should have received a copy of the GNU General Public License
% along with this document; if not, write to the Free Software Foundation,
% Inc., 59 Temple Place, Suite 330, Boston, MA 02111-1307, USA.
%
%
%
% vorwort.tex -> Vorwort
%
% Copyright (C) 2004 Joerg Meinhold
% 
% Dieses Dokument ist freie Software. Sie k�nnen es unter den Bedingungen
% der GNU General Public License, wie von der Free Software Foundation
% ver�ffentlicht, weitergeben und/oder modifizieren, gem�� Version 2
% der Lizenz.
%
% Die Ver�ffentlichung dieses Dokuments erfolgt in der Hoffnung, da�
% es Ihnen von Nutzen sein wird, aber OHNE IRGENDEINE GARANTIE, sogar
% ohne die implizite Garantie der MARKTREIFE oder der VERWENDBARKEIT
% F�R EINEN BESTIMMTEN ZWECK. Details finden Sie in der GNU General
% Public License.
%
% Sie sollten eine Kopie der GNU General Public License zusammen mit
% diesem Dokument erhalten haben. Falls nicht, schreiben Sie an die
% Free Software Foundation, Inc., 59 Temple Place, Suite 330, Boston,
% MA 02111-1307, USA.
%
%%%%%%%%%%%%%%%%%%%%%%%%%%%%%%%%%%%%%%%%%%%%%%%%%%

\chapter{Vorwort}

Der GNU Emacs ist einer der m�chtigsten, wenn nicht \emph{der}
m�chtigste Editor, den es heutzutage gibt. Er ist nicht nur ein
Editor, sondern er stellt unter anderem auch eine komplette
Arbeitsumgebung zur Verf�gung. Emacs besitzt z.B. einen Dateimanager,
kann interaktiv Unix- oder Linux-Shell-Befehle ausf�hren, unterst�tzt
das Schreiben in fast allen Programmiersprachen. Man kann aus Emacs
heraus Quelltexte kompilieren. Emacs ist auch ein ein News- und
E-Mail-Client, er kann sogar Internet-Seiten anzeigen und man kann ihn
als FTP-Client verwenden. Nat�rlich kann man mit den entsprechenden
Zusatzprogrammen (z.B. LaTeX) auch hochwertige Textdokumente
erstellen, ja es gibt sogar ein kleines Hilfsprogramm, mit dem man
ASCII-Zeichnungen anfertigen kann.

Aufgrund dieses gro�en Funktionsumfanges sagt man auch scherzhaft,
der Emacs kann alles, nur leider nicht Kaffeekochen.

\section*{Warum dieses Buch?}

Emacs ist nat�rlich sehr gut dokumentiert. Es gibt eine hervorragende
Online-Hilfe und im Internet gibt es viele Anleitungen zu den
verschiedensten Funktionen von Emacs. Der Haken an der Sache ist, das
meiste davon ist in Englisch.

Es existieren eine ganze Reihe von deutschen Einf�hrungstexten
\footnote{Dem Emacs ist standardm��ig ein deutsches Tutorial in der
  Hilfe beigef�gt, welches sehr gut f�r den Einstieg in Emacs geeignet
  ist} zum Emacs, auch gibt es in den meisten allgemeinen Linuxb�chern ein
Kapitel �ber den Emacs. M�chte man dagegen �ber bestimmte
Funktionalit�ten von Emacs mehr wissen, bleibt einem oft nur die m�hevolle
Suche im Internet, wo es durchaus einige gute deutsche Texte zu
bestimmten Funktionen gibt oder man k�mpft sich durch die englische
Dokumentation.

Dieses Buch auf deutsch soll helfen, den Emacs und seine M�glichkeiten 
umfassender kennenzulernen und dabei die (meiner
Meinung nach) vorhandene L�cke zwischen den Einf�hrungstexten und den
Texten zu den speziellen Funktionen zu schlie�en. Dabei werde ich
mich an dem sehr guten, englischen Buch "`Learning GNU Emacs - Second
Edition"' von Debra Cameron, Bill Rosenblatt und Eric Raymond orientieren.

\subparagraph{Achtung!} \textbf{Dieses Buch schreibe ich in meiner Freizeit
und wird daher eine sehr langfristige Sache. Besonders jetzt im
Anfangsstadium wird es sehr unvollst�ndig sein! F�r die "`Fachleute"'
- es ist absolutes Alphastadium.}

\section*{Lizenz}

Dieses Dokument steht wie der GNU Emacs selber unter der 
General Public License, Version 2 (GPL). Sie finden die offizielle
englische Version  und eine deutsche �bersetzung im Anhang. Wenn Sie
also zuk�nftig Teile meiner Texte verwenden m�chten, so beachten Sie
dabei bitte genau die Bestimmungen der GPL.



%%% Local Variables: 
%%% mode: latex
%%% TeX-master: t
%%% End: 
