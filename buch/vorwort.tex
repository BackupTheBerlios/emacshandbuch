%% Emacshandbuch in deutsch
%%
%% Vorwort
%%
%% $Revision: 1.1 $
%% $Id: vorwort.tex,v 1.1 2004/04/19 20:58:19 spica Exp $
%%
%%%%%%%%%%%%%%%%%%%%%%%%%%%%%%%%%%%%%%%%%%%%%%%%%%
%
% vorwort.tex -> Preface File
%
% Copyright (C) 2004 Joerg Meinhold 
%
% This document is free software; you can redistribute it and/or modify
% it under the terms of the GNU General Public License as published
% by the Free Software Foundation; either version 2 of the License.
%
% This document is distributed in the hope that it will be useful, but
% WITHOUT ANY WARRANTY; without even the implied warranty of MERCHANTABILITY
% or FITNESS FOR A PARTICULAR PURPOSE. See the GNU General Public License
% for more details.
%
% You should have received a copy of the GNU General Public License
% along with this document; if not, write to the Free Software Foundation,
% Inc., 59 Temple Place, Suite 330, Boston, MA 02111-1307, USA.
%
%
%
% vorwort.tex -> Vorwort
%
% Copyright (C) 2004 Joerg Meinhold
% 
% Dieses Dokument ist freie Software. Sie k�nnen es unter den Bedingungen
% der GNU General Public License, wie von der Free Software Foundation
% ver�ffentlicht, weitergeben und/oder modifizieren, gem�� Version 2
% der Lizenz.
%
% Die Ver�ffentlichung dieses Dokuments erfolgt in der Hoffnung, da�
% es Ihnen von Nutzen sein wird, aber OHNE IRGENDEINE GARANTIE, sogar
% ohne die implizite Garantie der MARKTREIFE oder der VERWENDBARKEIT
% F�R EINEN BESTIMMTEN ZWECK. Details finden Sie in der GNU General
% Public License.
%
% Sie sollten eine Kopie der GNU General Public License zusammen mit
% diesem Dokument erhalten haben. Falls nicht, schreiben Sie an die
% Free Software Foundation, Inc., 59 Temple Place, Suite 330, Boston,
% MA 02111-1307, USA.
%
%%%%%%%%%%%%%%%%%%%%%%%%%%%%%%%%%%%%%%%%%%%%%%%%%%

\chapter{Vorwort}

Der GNU Emacs ist einer der m�chtigsten, wenn nicht \emph{der}
m�chtigste Texteditor, den es heutzutage gibt. Er ist nicht nur ein
Text-Editor, sondern er stellt unter anderem auch eine komplette
Arbeitsumgebung zur Verf�gung. Emacs besitzt z.B. einen Dateimanager,
kann interaktiv Unix- oder Linux-Shell-Befehle ausf�hren, unterst�tzt
das Schreiben in fast allen Programmiersprachen. Man kann aus Emacs
heraus Quelltexte kompilieren. Emacs ist auch ein ein News- und
E-Mail-Client, er kann sogar Internet-Seiten anzeigen und man kann ihn
als FTP-Client verwenden. Nat�rlich kann man mit den entsprechenden
Zusatzprogrammen (z.B. LaTeX) auch hochwertige Textdokumente
erstellen, ja es gibt sogar ein kleines Hilfsprogramm, mit dem man
ASCII-Zeichnungen anfertigen kann.

Aufgrund dieses gro�en Funktionsumfanges sagt man auch scherzhaft,
der Emacs kann alles, nur leider nicht Kaffeekochen.

\section*{Warum dieses Buch?}

Emacs ist nat�rlich sehr gut dokumentiert. Es gibt eine hervorragende
Online-Hilfe und im Internet gibt es viele Anleitungen zu den
verschiedensten Funktionen von Emacs. Der Haken an der Sache ist, das
meist davon gibt es nur in Englisch.

